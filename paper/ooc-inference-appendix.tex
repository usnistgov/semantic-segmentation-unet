%\def\year{2020}\relax
%
%\documentclass[letterpaper]{article} % DO NOT CHANGE THIS
%\usepackage{aaai20}  % DO NOT CHANGE THIS
%\usepackage{times}  % DO NOT CHANGE THIS
%\usepackage{helvet} % DO NOT CHANGE THIS
%\usepackage{courier}  % DO NOT CHANGE THIS
%\usepackage[hyphens]{url}  % DO NOT CHANGE THIS
%\usepackage{graphicx} % DO NOT CHANGE THIS
%\urlstyle{rm} % DO NOT CHANGE THIS
%\def\UrlFont{\rm}  % DO NOT CHANGE THIS
%\usepackage{graphicx}  % DO NOT CHANGE THIS
%\frenchspacing  % DO NOT CHANGE THIS
%\setlength{\pdfpagewidth}{8.5in}  % DO NOT CHANGE THIS
%\setlength{\pdfpageheight}{11in}  % DO NOT CHANGE THIS
%
%\usepackage{amsmath}
%%\usepackage{amssymb}
%\usepackage[binary-units]{siunitx}
%
%
%
%\begin{document}


\section{Appendix A: Derivation of Simplified Radius Formula}

Let us assume that in the entire U-Net architecture the kernel size is constant $k_{c} = k = const$ and each level has the same number of convolutional layers on both decoder and encoder sides $n_{l} = n = const$. If these constraints are satisfied, then the general formula for determining radius can be simplified as follows:

\begin{equation}
\begin{aligned}
Radius = \sum_{c=1}^{N} 2^{l_c} \lfloor \frac{k_c}{2} \rfloor \\
= \lfloor \frac{k}{2} \rfloor \times \sum_{c=1}^{N} 2^{l_c} \\
= \lfloor \frac{k}{2} \rfloor \times ( 2 \times \sum_{m=0}^{M-1} (2^{m} \times n) + 2^{M} \times n )  \\
= \lfloor \frac{k}{2} \rfloor \times n \times ( 2 \times \frac{1 \times (1 - 2^M) }{ 1- 2} ) + 2^{M} )  \\
= \lfloor \frac{k}{2} \rfloor \times n \times (3 \times 2^{M} - 2) 
\end{aligned}
\end{equation}

where $M$ is the maximum U-Net level $M = \max_{\forall c}\{ l_{c} \}$ .

For U-Net architecture, the parameters are $k = 3$, $n=2$ and $M=4$, and the equation yields $Radius=92$.

For DenseNet architecture, the parameters are $k = 3$, $n=4$ and $M=5$, and the equation yields $Radius=376$. This value differs by one from the value computed according to Equation 1 because the DenseNet has asymmetry between the first encoder layer with a kernel size $k = 3$ and the last decoder layer with a kernel size $k = 1$.   



\section{Appendix B: Example U-Net Radius Calculation}

Following Equation \ref{eq:radius} for U-Net results in a required radius of 92 pixels in order to provide the network with all of the local context it needs to predict the outputs correctly. With $k = 3$, $\lfloor \frac{k_c}{2} \rfloor$ reduces to $\lfloor \frac{3}{2} \rfloor = 1$. The radius computation for U-Net thus reduces to a sum of $2^{l_c}$ terms for each convolutional layer encountered along the longest path from input to output as shown in Equation \ref{eq:unet-radius}. 

\begin{equation}
Radius = \sum_{c=1}^{18} 2^{l_c}
\label{eq:unet-radius}
\end{equation}

By substituting the level numbers for each convolutional layer from 1 to 18 as shown in Equation \ref{eq:unet-radius-num}, one obtains the minimum radius value of 92 pixels. 

\begin{equation}
\begin{small}
\begin{aligned} 
l_c = \{0, 0, 1, 1, 2, 2, 3, 3, 4, 4, 3, 3, 2, 2, 1, 1, 0, 0\} \\
92 = 2^0 + 2^0 + 2^1 + 2^1 + 2^2 + 2^2 + 2^3 + ...
\end{aligned}
\end{small}
\label{eq:unet-radius-num}
\end{equation}

Similarly, according to Equation \ref{eq:radius2}, the calculation simplifies to:
\begin{equation}
\begin{aligned} 
M=\max_{\forall c} { l_c } = 4\\
k_c = k = 1  \\
n_l = n = 2 \\
92 = 1 \times 2 \times (3 \times 2^4 - 2)
\end{aligned}
\label{eq:unet-radius-num2}
\end{equation}



%\end{document}
